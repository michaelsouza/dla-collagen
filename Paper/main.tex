\documentclass{article}
\usepackage{graphicx} % Required for 
\usepackage{natbib} % For bibliography
\usepackage{float} % Add this in your preamble


\title{paper collagen}
\author{robert.tavares}
\date{November 2023}

\begin{document}

\maketitle

\section{Introduction}

O colágeno tipo I é uma proteína abundante no corpo humano, de forma que compõe $90\%$ dele. Ele está presente na matriz extracelular bem como em estruturas como pele, osso, tendão e córnea \cite{RicoLlanos2021, Silver2018}.
\\

O colágeno tipo I é uma molécula composta por um tripla hélice de cadeias polipeptídicas(cadeias $\alpha$). Ele possui tamanhos típicos de $300$ nm de comprimento por $1,5$ nm de diâmetro e apresenta um formato tipo haste \cite{Gelse2003,Silver2018}. 
\\

As moléculas de colágeno tipo I possuem, intrisicamente, a informação necessária para se auto-organizar em estruturas mais complexas chamadas fibrilas. Esse processo, denominado fibrilo-gênese, ocorre o mediante a agregação de milhares dessas moléculas de forma escalonada por um período $D= 67$ nm, de modo que existem cinco posições possíveis para que ocorra a interação entre elas\cite{Zhu2018, KADLER1996}.  
\\

As fibrilas apresentam um formato alongado, com a sua região mais densa sendo a central e, consequentemente, as pontas afinadas\cite{Charvolin2019, KADLER1996}. As fibrilas apresentam comprimento típicos de $500 \mu m$ com $500 nm$ de diâmetro, além de serem constituídas por moléculas da ordem de $10^{7}$\cite{Parry1984}. 
\\


\section{Metodologia}


Em nosso trabalho, nos tivemos duas etapas desenvolvidas: Modelo para formação de fibrilas e Morfologia e propriedades mecânicas das fibrilas. 

\subsection{Formação de fibrilas}

Nos utilizamos um modelo baseado em DLA(difusion limited aggregation)\cite{Witten1983} em três dimensões para tentar simular a formação das fibrilas de colágeno. As moléculas são lançadas de uma distância $R$ da origem do nosso agregado e se difundem até encontrarem o mesmo ou atingirem uma distância  $2R$, de modo que a simulação é reiniciada caso isso ocorra. Ao atingir o agregado, a molécula é capturada apenas se ela estiver em uma posição que é um múltiplo de $D = 4$ com relação a uma molécula pertencente ao agregado\cite{Parkinson1995}. 
\\

Cada molécula é representada por um bastão de dimensão 1 x 18 x 1 e pode se mover entre os primeiros e segundos vizinhos no plano XZ, enquanto pode se mover para frente e para trás no eixo Y.
\\


O processo de formação das fibrilas é dirigido por uma força hidrofóbica devido a interação entre as moléculas, de modo que elas tentam minimizar sua superfície exposta para tal\cite{Kadler1987, Parkinson1995}. Para representar esse processo, utilizamos um algoritmo de rolamento sobre a superfície, que permite um bastão recém agregado explorar a superfície do agregado, mantendo seu y fixo, para encontrar um local que minimize sua superfície exposta\cite{GarcaRuiz1991}. Caso ocorra mais de um local, a primeira posição é mantida. Esse movimento é controlado pelo parâmetro $T_{s}$ que define o número de passos que a molécula tem para explorar a superfície do agregado\cite{Parkinson1995}.
\\

Com esse algorítimo, geramos $10$ fibrilas contendo $30.000$ bastões para diferentes valores de $T_{s}$ a fim de ver o efeito desse parâmetro na morfologia das fibrilas. 


\subsection{Propriedades das fibrilas}

Para analisar as propriedades mecânicas das fibrilas, utilizamos um modelo mecânico probabilístico, visto que as fibrilas geradas não possuem um carácter elástico para ser estudado como normalmente é feito\cite{Parkinson1997}. 

Para cada  amostra de um dado valor do parâmetro $T_{s}$, fizemos o corte de um tronco de dimensão 101 x 201 x 101 na região central das fibrilas afim de minimizar efeitos de borda. Depois, realizamos uma limpeza nesse tronco para eliminar moléculas que não estavam pertencentes ao esqueleto ativo\cite{Parkinson1997}.

Ao aplicarmos uma força no esqueleto, calculamos a pressão $\sigma$, onde a área é dada pelo número de elementos do esqueleto ativo em uma dada camada do tronco. No modelo mecânico estocástico, para cada molécula no agregado, determinamos uma probabilidade de remoção dada por:

\begin{equation}
    P_{R} = (\frac{<\sigma>}{N\sigma_{s}})^{m},
\end{equation}

\noindent onde $\sigma$ é a média da pressão que cada pedaço de uma molécula sente, $N$ é o número de ligações que uma dada molécula possui, assumimos que cada face em contato com uma molécula vizinha contribui com uma ligação, $\sigma_{c}$ a força da ligação entre as moléculas, que tomamos como unitária e m é um fator de amortecimento da energia\cite{Parkinson1997,2013}.

A simulação consiste em aplicarmos umas força no esqueleto ativo, com isso, calculamos as probabilidades de cada molécula ser removida e sortíamos um número para ver se ocorre a ruptura. Caso ocorra pelo menos uma única quebra, repetimos o procedimento para uma dada força até que não mais ocorra rupturas. Nesse ponto, incrementamos a força em meio e repetimos a simulação. A fibrila se quebra se ocorrer a existência de pelo menos uma camada vazia.

(fazer uma figura disso)

Com isso, realizamos, para cada fibrila com um determinado valor de $T_{s}$, mil experimentos. Guardamos informações dos valores, para cada força, do número removido do esqueleto, bem quantas restaram.

\section{Resultados}

\subsection{Morfologia das fibrilas}

Os agregados gerados pelo modelo apresentam uma morfologia fibrilar, com características relevantes de sua forma 
sendo determinadas pelo parâmetro \(T_{s}\). Podemos observar, na Figura \ref{R1}, a estrutura desses agregados 
para os valores de \(T_{s} = 2\), baixa difusão, e \(T_{s} = 10000\), alta difusão lateral sobre a superfície. As 
fibrilas com menor \(T_{s}\) apresentam uma forma mais aberta, enquanto para valores mais altos, observamos uma 
forma mais compacta e regular. A coloração indica o quão antiga uma molécula é no agregado, indo do azul escuro, 
mais antigas, para o amarelo, mais recentes. Nos agregados mais compactos, temos dificuldade em observar moléculas
mais antigas visto que essas estão muito no interior da estrutura. Para as mais abertas, temos uma maior facilidade
em observar moléculas mais antigas. Além disso, na visão lateral, observamos o comportamento alongado e com pontas 
afinadas, típico de fibrilas reais. 

\begin{figure}[H]
    \centering
    \includegraphics[width=\textwidth]{figures/fibrils.png}

    \caption{Visualização transversal e lateral das fibrilas geradas com o algoritmo de DLA contendo 30.000 moléculas.
    A coloração indica quão antigo é a molécula no agregado. Quanto mais pro azul escuro, 
    mais antigo no agregado, quanto mais para o amarelo, mais recente. 
    A) Fibrila gerada para $T_{s} = $ 2, baixa difusão. B) Fibrila gerada para $T_{s} = $ 10000, alta difusão.} 

    \label{R1}
\end{figure}

O comprimento, o diâmetro e a densidade da região central das fibrilas são características influenciadas pelo 
parâmetro \(T_{s}\). Na Tabela \ref{tab1}, podemos observar como essas dimensões se alteram, em média, com o 
aumento desse parâmetro. O comprimento e a densidade tendem a aumentar com o incremento de \(T_{s}\), enquanto 
o diâmetro tende a diminuir. Uma propriedade comum a essas medidas é que elas exibem um comportamento de 
estabilização à medida que nos aproximamos de \(T_{s} = 512\); a partir desse ponto, elas oscilam em torno de um 
valor médio. 

\begin{table}[H]
    \caption{.}

    \centering  % Mantém a tabela centralizada no texto
    \begin{tabular}{lccc}
    \hline
    \textbf{$T_{s}$} & \multicolumn{1}{c}{\textbf{Length(u.m)}} & \textbf{Ray(u.m)} & \textbf{Density(\% )} \\ \hline
    2                & 3668.36                                   & 32.40             & 0.17                  \\
    8                & 3695.16                                   & 28.68             & 0.25                  \\
    16               & 3764.27                                   & 24.63             & 0.34                  \\
    32               & 3808.68                                   & 21.67             & 0.46                  \\
    64               & 3891.56                                   & 17.6              & 0.57                  \\
    128              & 3928.6                                    & 16.06             & 0.62                  \\
    512              & 3913.24                                   & 14.07             & 0.66                  \\
    1024             & 3912.52                                   & 14.14             & 0.65                  \\
    4096             & 3892.28                                   & 14.06             & 0.66                  \\
    8192             & 3892.52                                   & 14.16             & 0.66                  \\
    10000            & 3917.16                                   & 13.94             & 0.65                  \\ \hline
    \multicolumn{1}{l}{Limit Upper} & 3905.54                    & 14.08             & 0.65                  \\ \hline
    \end{tabular}
    \label{tab1}  % Substitua 'meu_rótulo' pelo rótulo desejado para a referência cruzada
\end{table}


Outra característica desses agregados é a relação linear entre a massa e a distância até as pontas. Na Figura 
\ref{R2}, observamos que, independentemente do valor de \(T_{s}\), todos os agregados exibem esse comportamento. 
Tal característica é recorrentemente observada tanto em fibrilas reais quanto nas simuladas com este modelo 
\cite{Parkinson1995,Kadler1987}. 


\begin{figure}[H]
    \centering
    \includegraphics[width=\textwidth]{figures/tips.png}

    \caption{A quantidade de partículas por seção das fibrilas geradas segue uma relação linear da distancia em que 
    medimos para a ponta. Nos observamos que, partindo de uma ponta, a massa cresce linear até bem proximo da região
    central da fibrila. A medida que nos afastamos dessa região, a massa decaí linearmente. Esse comportamento foi 
    observado para todas as fibrilas, indicando que o parâmetro $T_{s}$ não tem efeito sobre essa característica.} 

    \label{R2}
\end{figure}


Analisando a seção transversal das fibrilas, conforme ilustrado na Figura \ref{R3}, observamos que o aumento do 
parâmetro \(T_{s}\) resulta na diminuição dos espaços vazios dentro da seção, levando à formação de agregados mais 
compactos e quase completamente preenchidos. Devido a essa característica progressiva em função do parâmetro 
\(T_{s}\), calculamos a dimensão fractal das seções e constatamos que, à medida que \(T_{s}\) aumenta, ocorre um 
incremento no valor médio da dimensão fractal da seção até atingir uma saturação. Na Figura \ref{R4}, é evidente 
que para valores mais baixos de \(T_{s}\), a dimensionalidade é próxima da observada em agregados gerados pelo 
modelo de Agregação Limitada por Difusão (DLA) \cite{Witten1983}, que é de 1.71, remetendo ao nosso modelo de 
formação. Enquanto isso, para valores mais elevados de \(T_{s}\), a dimensão fractal tende a estabilizar em valores 
próximos a 1.93, que se assemelha muito à dimensão euclidiana para objetos bidimensionais. 

\begin{figure}[H]
    \centering
    \includegraphics[width=\textwidth]{figures/cs_all.png}
    \caption{Variação da forma da seção transversal das fibrilas para diferentes valores de $T_{s}$.} 
    \label{R3}
\end{figure}

\begin{figure}[H]
    \centering
    \includegraphics[width=\textwidth]{figures/dim_frac.png}

    \caption{ } 

    \label{R4}
\end{figure}


\subsection{Propriedades mecânicas}

Para compreender como nosso agregado responde à aplicação de uma força axial, avaliamos como o número de 
moléculas na fibrila varia com a força. Na Figura \ref{R5}, apresentamos a curva de stress-strain para agregados 
gerados com diferentes valores de \(T_{s}\). Observamos que o aumento desse parâmetro influencia no incremento da 
tensão máxima de ruptura. No entanto, a partir de \(T_{s} = 512\), esses valores se tornam bastante próximos e as 
curvas começam a se sobrepor. Comumente, esperaríamos que essas curvas crescessem de forma mais abrupta até o valor
máximo, o que não é observado aqui. Esse comportamento mais suave até o valor limite é uma consequência do módulo 
de Weibull que utilizamos no modelo. Para valores baixos, a curva resultante é não determinística 
\cite{Parkinson1997}. A tensão máxima suportada que encontramos foi de 45,6 MPa, valor muito próximo ao encontrado 
por Yang et al. \cite{YANG2012148} ao analisar fibrilas reconstituídas do tendão de Aquiles bovino purificado. 
Em contraste, no trabalho de Yamamoto \cite{Noritaka}, foram utilizadas fibrilas isoladas do fascículo dos tendões 
da cauda de ratos, obtendo um valor de 100 \(\pm\) 32 MPa. A discrepância em relação ao nosso valor pode estar 
associada à dimensão das fibrilas utilizadas por ele, que apresentavam um diâmetro significativamente maior do que 
as modeladas neste trabalho.

\begin{figure}[H]
    \centering
    \includegraphics[width=\textwidth]{figures/stress_strain.png}

    \caption{Tensão em função da deformação da fibrila para diferentes valores de $T_{s}$.} 

    \label{R5}
\end{figure}

Na Figura \ref{R6}, podemos observar como os valores de tensão máxima suportada variam com um parâmetro 
importante das fibrilas, a densidade. Escolhemos essa análise visto que o comportamento de \(\sigma\) e da 
densidade, \(\rho\), exibem formas semelhantes quando analisados em função de \(T_{s}\). Identificamos um 
comportamento exponencial no aumento da tensão máxima até um limite superior de 44,1 MPa. A saturação da 
densidade ocorre em 65\%, valor este próximo ao encontrado por Parkinson et al.\cite{Parkinson1995} com esse 
modelo, porém um pouco abaixo do determinado por Katz et al.\cite{KATZ1973351}, que calculou experimentalmente 
cerca de 80\% do espaço disponível ocupado para as fibrilas de colágeno. Com base nessa característica de 
saturação, consideramos que, dado o custo computacional elevado, este modelo pode ser executado, para os parâmetros 
que inicialmente utilizamos, com \(T_{s}=512\), uma vez que nesse ponto já obtemos fibrilas com os valores de 
interesse médios equivalentes para valores superiores desse parâmetro. 



\begin{figure}[H]
    \centering
    \includegraphics[width=\textwidth]{figures/sigma_rho.png}

    \caption{Tensão crítica em função da densidade. Observamos que esse valor cresce exponencialmente com a 
    densidade até ambos os parâmetros atingirem um limite superior.} 


    \label{R6}
\end{figure}

Ao analisar como o processo de ruptura ocorre no nosso modelo, bem como em outros, constatamos que a 
redistribuição de tensão, para uma mesma força, pode levar à ocorrência de rupturas em cascata. Na Figura 
\ref{R7}, apresentamos a distribuição das avalanches de ruptura em função do seu tamanho. É possível observar a 
existência de duas ordens de grandeza bem definidas; após isso, os dados são afetados pelo efeito de tamanho 
finito. Analisando a região de interesse, até próximo de \(10^{2.5}\), conseguimos determinar o expoente das leis 
de escala de modo que eles aumentam com o crescimento de \(T_{s}\). Este comportamento é compreensível ao 
considerarmos que a densidade aumenta com este parâmetro, resultando em mais moléculas para contribuir com os 
tamanhos das avalanches. Assim, temos que as avalanches durante o processo de ruptura são independentes do tamanho 
do sistema; contudo, elas são influenciadas pelo quão compacta é a fibrila, com o expoente \(\gamma\) variando de 
-1.94 até -2.60. 


\begin{figure}[H]
    \centering
    \includegraphics[width=\textwidth]{figures/ava.png}

    \caption{Distribuição das avalanches de ruptura em função do seu tamanho para diferentes valores de $T_{s}$.
    Podemos observar um comportamento em lei de potencia com expoentes $\gamma$ bem definidos para todos valores do 
    parâmetro. Os expoentes variam de -1.94 a -2.60.} 

    \label{R7}

\end{figure}

\section*{Conclusões}







\bibliographystyle{plain} % Set the bibliography style
\bibliography{ref} % Include your .bib file (without the .bib extension)


    
\end{document}
