% \documentclass[lineno,twocolumn,endfloat,biblatex]{biophys-new}

\documentclass{biophys-new}
\usepackage[utf8]{inputenc}
\usepackage{graphicx}
\usepackage[colorlinks,allcolors=cyan!70!black]{hyperref}

% Uncomment if using biblatex
% \addbibresource{sample.bib}

\usepackage{lipsum}

\title{Scaling behaviors in simulated collagen fibrils}
\runningtitle{Biophysical Journal Template} %% For page header

\author[1]{Robert Bertoldo Tavares}
\author[2]{Michael Ferreira de Souza}
\author[3]{B\'ela Suki}
\author[1,2,*]{Ascânio Dias Araújo}
\runningauthor{Tavares et al.} %% For page header

\affil[1]{UFC, Departamento de Física, Universidade Federal do Ceará, Fortaleza, Ceará, Brasil}
\affil[2]{UFC, Departamento de Estatística e Matemática Aplicada, Universidade Federal do Ceará, Fortaleza, Ceará, Brasil}
\affil[3]{BU, Department of Biomedical Engineering, Boston University, Boston, MA, USA}
\corrauthor[*]{ascanio@fisica.ufc.br}

% \papertype{Letters}
\papertype{Article}
% \papertype{Computational Tools}


\begin{document}


\begin{frontmatter}

	\begin{abstract}
		Collagen fibrils are hierarchical protein assemblies that provide tensile strength and structural support to mammalian tissues. Understanding the relationship between assembly kinetics, structural organization, and mechanical properties is fundamental to soft matter physics and biomaterials science. Here, we investigate collagen fibril formation and mechanical failure through complementary computational models: a diffusion-limited aggregation (DLA) algorithm with lateral surface diffusion that simulates the entropy-driven self-assembly process, and a probabilistic fracture mechanics model. Fibril assembly is governed by a surface diffusion parameter $T_s$ that represents the molecular mobility and rearrangement capability of collagen molecules after initial attachment. Increasing $T_s$ drives a morphological transition from open, fractal structures (fractal dimension $D_f \approx 1.71$, characteristic of DLA) to dense, compact fibrils ($D_f \approx 1.96$, approaching Euclidean space-filling). Critically, we find that $D_f$ is a robust predictor of both tensile strength and failure mode: higher $D_f$ corresponds to enhanced load-bearing capacity through increased connectivity of the active skeleton. Mechanical failure proceeds via avalanche-like rupture events whose size distributions follow power-law scaling, with exponents that depend systematically on $T_s$. Notably, these scaling exponents exhibit two distinct linear regimes when plotted against $\log(T_s)$, indicating a transition in the dominant failure mechanism between diffusion-limited and compact packing regimes. These findings establish fractal dimension as a quantitative link between assembly conditions and mechanical resilience, providing a framework for understanding how molecular-scale processes during self-assembly determine the emergent failure mechanics of hierarchical biomaterials.
	\end{abstract}

	\begin{sigstatement}
		Understanding collagen fibril formation and damage is crucial, because collagen fibrils are fundamental to the integrity of virtually all tissues. This study reveals that the fracture process in simulated collagen fibrils is governed by avalanches and power laws with well-defined exponents. More importantly, the results demonstrate that the fractal dimension of fibril cross-section is a robust indicator of both the failure mechanics and the failure mode of collagen fibrils.
	\end{sigstatement}
\end{frontmatter}

\section*{Introduction}

Type I collagen is the principal structural protein of the mammalian extracellular matrix (ECM), providing essential mechanical integrity to tissues such as skin, tendon, bone, cartilage, and the cornea \cite{Silver2018,DavisonKotler2019,Tresoldi2013,Fedarko2014, RicoLlanos2021,Chen2015,Suki2021}. These hierarchical assemblies are critical for elasticity, tensile strength, and resilience against physical stress \cite{Silver2018,Amirrah2022}. Individual collagen molecules, characterized by a high aspect ratio ($\approx 300~\text{nm} \times 1.5~\text{nm}$) \cite{Gelse2003, Silver2018}, self-assemble into fibrils that can reach lengths of $500~\mu\text{m}$ and diameters of $500~\text{nm}$ \cite{Charvolin2019,Kadler1996, Parry1984}. These fibrils further aggregate into fibers ($1$--$20~\mu\text{m}$ diameter), forming a multiscale network capable of storing and transmitting mechanical energy \cite{Silver2008,Suki2021}.

Characterizing the mechanical behavior of collagen fibrils, particularly at the nanoscale, presents significant challenges \cite{Nalbach2022}. Experimental techniques such as atomic force microscopy (AFM) and micro-mechanical actuation have successfully probed the tensile properties of isolated fibrils. Van der Rijt et al. used AFM to measure the mechanical properties of tendon fibrils, evaluating their tensile behavior until failure \cite{vanderRijt2006}, while Svensson et al.~\cite{Svensson2018} developed a piezoelectric actuator-based technique to measure stress and strain during uniaxial stretching. However, resolving the collective molecular dynamics that govern failure remains difficult due to experimental resolution limits, particularly when characterizing small-scale forces and intermolecular interactions \cite{Andriotis2023}. Consequently, computational modeling has become indispensable for exploring the relationship between intrafibrillar organization and mechanical response under a wider range of conditions.

Several theoretical frameworks have been developed to address fibril mechanics. Buehler et al. introduced coarse-grained molecular dynamics to model large-scale deformation, while Depalle et al. investigated the role of mineralization in stabilizing fibril elasticity \cite{Buehler2006a,Buehler2006b,Buehler2008a,Buehler2008b,Depalle2016}. Other stochastic approaches have examined how enzymatic degradation alters tissue mechanics \cite{Arau2011} and how morphological features like waviness influence fiber modulus \cite{Deng2024}. Despite these advances, the influence of fractal organization on mechanical resilience and the statistical dynamics of rupture—specifically the role of failure avalanches—remains largely unexplored.

In this work, we investigate the interplay between fractal structure and failure mechanics in collagen fibrils. We employ a modified Diffusion-Limited Aggregation (DLA) model with surface diffusion to generate fibrils with tunable morphologies \cite{Parkinson1995}. By coupling this formation model with a probabilistic fracture simulation, we show that the surface diffusion parameter $T_s$ dictates both the fractal dimension and the mechanical strength of the fibril. Furthermore, we reveal that rupture defines a critical process characterized by avalanche-like failure events, which follow power-law scaling distributions reminiscent of self-organized critical systems.

\subsection*{Fibril Formation Model}

We modeled collagen fibril formation using a three-dimensional diffusion-limited aggregation (DLA) algorithm on a cubic lattice. Collagen molecules were represented as rigid rods (rectangular bars) with dimensions $L_x = 1$, $L_y = 18$, and $L_z = 1$ in lattice units (l.u.), with $L_y$ defining the characteristic length scale of the model.

The aggregation begins by placing the first molecule at the lattice center. Subsequent molecules are released one at a time from random positions on a sphere of radius $R$ centered at the origin. The radius $R$ is updated to track the maximum extent of the growing aggregate along the $y$-axis. Each molecule then performs a random walk on the lattice until one of the following conditions is met:

\begin{enumerate}
	\item The molecule attaches to the existing aggregate;
	\item The molecule is discarded if its distance from the origin exceeds $2R$.
\end{enumerate}

After each attachment, $R$ is recalculated. For simplicity, molecular rotation during diffusion is not allowed. This cycle of release, diffusion, and attachment continues until the aggregate reaches the desired size, limited here to 30,000 molecules.

Diffusion is implemented as a three-dimensional random walk on the cubic lattice. In the $x$-$z$ plane, molecules can move to first-neighbor sites ($1~\text{l.u.}$) or second-neighbor sites ($\sqrt{2}~\text{l.u.}$, i.e., diagonal moves). Along the $y$-axis, movement is restricted to first-neighbor sites ($1~\text{l.u.}$). In real collagen fibrillogenesis, lateral aggregation occurs with a characteristic axial spacing of $67~\text{nm}$ \cite{Kadler1996}, which constrains the positions where new molecules can be incorporated. In our model, molecular attachment is restricted to multiples of $4~\text{l.u.}$ along the $y$-axis \cite{Parkinson1995}.

The formation of real fibrils is driven by electrostatic forces \cite{Parkinson1995} that favor lateral association and minimize exposed surface area \cite{Kadler1987,Parkinson1995}. To mimic this relaxation, we implemented a lateral surface-diffusion step after attachment: a newly incorporated molecule performs a random walk restricted to the $x$-$z$ plane (no movement along the $y$-axis), using the same step rules described above. The molecule is allowed up to $T_s$ diffusion attempts to search the local surface and is placed at the position that minimizes its exposed area; if multiple positions are equivalent, the first one encountered is retained \cite{Garci1991}. Using this algorithm, we generated ensembles of fibrils (each containing at least 30,000 rods) for different values of $T_s$ to examine how post-attachment mobility controls fibril compactness, structure, and subsequent mechanical response.

\begin{figure}[!htb]
	\centering
	\includegraphics[width=0.7\textwidth]{Figures/figure_1.pdf}
	\caption{Representation of portions of fibrils generated with the DLA algorithm containing 30,000 molecules. Panels $(A)$ and $(B)$ show typical fibrils obtained with $T_s = 2$ (short-term diffusion) and $T_s = 8192$ (long-term diffusion), respectively. Colors indicate the relative time of attachment, from early (blue) to late (red).}
	\label{fig_1}
\end{figure}

The generated fibrils exhibit complex, heterogeneous packing with internal voids whose prevalence depends strongly on $T_s$ (Fig.~\ref{fig_1}). For $T_s = 2$, the fibril displays an open architecture with visible gaps between molecular clusters. In contrast, for $T_s = 8192$, the structure becomes more compact and homogeneous, with substantially reduced void space.

\subsection*{Radial Expansion of the Fibril Core}

Fibril growth occurs more strongly in the $y$-direction. We aimed to study the density of the cross-section in the $x$-$z$ plane. Using the original simulation for this purpose would be computationally prohibitive. Therefore, alternatively, we isolated a trunk in the $y$-direction from $-100$ to $100$ and used a cylindrical surface instead of a sphere for particle launching. We repeated the deposition until the average radius of the cross-sections ($y=-100, -99, \dots, 100$) reached $100~\text{l.u.}$. Molecules diffused inward until binding to the aggregate, following the same diffusion, attachment, and surface-diffusion rules described above. Fig.~\ref{fig_2} shows the resulting projected central segments in the $x$-$z$ plane for different values of $T_s$.

\begin{figure}[!htb]
	\centering
	\includegraphics[width=0.7\textwidth]{Figures/figure_2.pdf}
	\caption{Representative examples of projected central segments ($x$-$z$ plane) for different values of $T_s$: (a) $T_s = 2$, (b) $T_s = 64$, (c) $T_s = 512$, and (d) $T_s = 8192$. The progression illustrates a clear transition from sparse, irregular morphology at low $T_s$ to dense, radially symmetric packing at high $T_s$. The color gradient (blue to red) represents the temporal sequence of molecular incorporation, from early-attached to recently added molecules.}
	\label{fig_2}
\end{figure}

Fig.~\ref{fig_2} illustrates how $T_s$ controls radial organization. At low $T_s$ (Fig.~\ref{fig_2}a), the projected cross-section is open and branched. As $T_s$ increases, growth becomes progressively more compact and regular, reaching the most uniform configuration at $T_s = 8192$ (Fig.~\ref{fig_2}d). Because the number of intermolecular contacts increases with local packing density, this transition suggests enhanced connectivity and mechanical stability at larger $T_s$ \cite{Mohammadkhah2023}.

To quantify how $T_s$ modulates fibril structure, we estimated the fractal dimension of the central segments. For each $T_s$, we generated 50 fibrils using the first model (spherical launch), isolated the central region of each fibril, and performed radial growth using the second model (cylindrical launch). From each grown region, we extracted 11 cross-sections from $y=-90$ to $y=90$ with a spacing of $18~\text{l.u.}$, yielding a total of 550 cross-sections per $T_s$ value. For each cross-section, we computed the center of mass and the individual maximum radius enclosing all particles; $R_{\text{max}}$ was then defined as the largest of these 550 individual radii. The mass $m(R)$ was measured as the number of particles within a circular region of radius $R$ centered at the cross-section's center of mass. For each value of $R$ spanning from $R_{\text{min}} = 5~\text{l.u.}$ to $R_{\text{max}}$, we computed the mean mass $\langle m(R) \rangle$ by averaging over all 550 cross-sections \cite{Vicsek1991}. The fractal dimension $D_f$ was then extracted from the mass--radius scaling relation \cite{Giordano2012}:

\begin{equation}
	\langle m(R) \rangle \sim R^{D_{f}},
	\label{eq1}
\end{equation}

by performing a linear fit on a log-log plot of $\langle m(R) \rangle$ versus $R$.

As shown in Fig.~\ref{fig_3}, the fractal dimension $D_f$ depends strongly on $T_s$, providing a quantitative measure of the morphological transition observed in Fig.~\ref{fig_2}. For $T_s = 2$, we find $D_f = 1.708 \pm 0.005$, close to the characteristic value of two-dimensional diffusion-limited aggregation ($D_f \approx 1.71$) \cite{Witten1983}. As $T_s$ increases, $D_f$ rises and saturates at $D_f = 1.963 \pm 0.001$ for $T_s = 8192$, approaching the Euclidean limit $D_f = 2.0$ expected for a compact two-dimensional disk. Thus, increasing post-attachment mobility drives a crossover from diffusion-limited, fractal packing to nearly space-filling cross-sections.

\begin{figure}[ht]
	\centering
	\includegraphics[width=0.7\textwidth]{Figures/figure_3.pdf}
	\caption{Average fractal dimension $D_f$ as a function of the diffusion parameter $T_s$ on a linear--log scale. Error bars are shown in red. The fractal dimension increases from $D_f = 1.708 \pm 0.005$ at $T_s = 2$ to $D_f = 1.963 \pm 0.001$ for $T_s = 8192$.}
	\label{fig_3}
\end{figure}


\subsection*{Mechanical Model for Fibril Rupture}

Since the collagen molecules are modeled as rigid rods that do not undergo continuous deformation under stress prior to rupture, a deterministic fracture simulation is not feasible. We therefore employed a probabilistic framework, a conventional method for modeling failure in disordered systems \cite{z2019, Gilabert1987, Noguchi2024}. In this framework, failure occurs through the random removal of bonds between fundamental units, with the probability of bond breakage depending on whether the applied force exceeds a threshold value. Thus, entire molecules (rigid rods that constitute the primary structural units) are removed from the fibril according to a probability function. The model was implemented according to the following steps:
$(i)$ Identify the trunk (the force-carrying core at the left end) in the middle of the fibril with dimensions $S_x = 17 $, $S_y = 201 $ and $S_z = 17$ l.u. \cite{Parkinson1997}. Inevitably, the removal of part of the fibril for rupture analysis results in regions that remain part of the fibril but consist of disconnected molecules (dangling ends), which do not contribute to force transmission \cite{Herrmann1984}. This detail leads us to our next step, $(ii)$ identifying the force-carrying active skeleton (backbone) prior to applying tensile tests \cite{Moreira2012}.

In order to do that, we mark all molecules that have a segment in the first cross-sectional area, designated by $y = 1$, as active. The identification process proceeds for each cross-sectional area along the isolated trunk, where each molecule belongs to the cross-sectional area that is marked. This  process continues until it reaches the last cross-sectional area located at $y = S_y$. Once the last cross-sectional area is analyzed, the procedure is reversed, starting from $y = S_y $ and moving back toward $y = 1$, considering only those molecules previously marked as active. At the end, only molecules with structural connectivity remain, forming the final backbone \cite{Parkinson1997}. The steps of this process are illustrated in Fig.~\ref{fig_4}.

\begin{figure}[ht]
	\centering
	\includegraphics[width=\textwidth]{Figures/figure_4.pdf}
	\caption{(A) A typical two-dimensional visualization of the fibril trunk, which is selected and isolated from the entire fibril to identify the force-carrying active skeleton. The circle highlights the region where the identification process starts. (B) Step-by-step procedure for identifying the active skeleton, from left to right. As we proceed, the active portion of the trunk is highlighted in black. This continues until the opposite end is reached and the entire backbone is identified. At the end of this process, only the structure referred to as the conductive skeleton remains, allowing us to proceed with the analysis of the fracture process. }
	\label{fig_4}
\end{figure}

With the backbone identified, we can simulate a tensile test by incrementally applying an external force $F$ along the main axis $(y)$. To translate this force into local stress, we evaluate at each cross-section along the backbone the effective support area, i.e., the area occupied by the molecules. Collagen molecules are modeled as rigid rods of length $L_y = 18$, therefore, each cross-section containing a molecule includes a one-unit-long discrete segment, and each of these segments is assumed to have unit-area \cite{Parkinson1995}. Thus, for cross-section $i$, the effective force-carrying area is the number of segments $N(i)$ that are part of the skeleton. For simplicity, we assume that $F$ is uniformly distributed among these units, the local stress at a cross-section $i$ is given by \cite{Parkinson1997}:

\begin{equation}
	\sigma(i) = \dfrac{F}{N(i)}.
	\label{eq2}
\end{equation}

For simplicity, we assume that structural failure occurs through the breakage of intermolecular bonds, leading to the removal of entire molecules from the backbone—not by isolated segments. This assumption is based on the fact that the internal bonds between segments that compose the molecule exceed the strength of intermolecular forces \cite{Parkinson1997}. In other words, we explicitly assume that intermolecular bonds fail but the collagen molecule itself never ruptures; instead, it is removed as a whole intact unit. To decide probabilistically on this removal, we first calculate the average stress acting on each molecule. Since the molecule extends over multiple cross-sections, the average stress on a molecule, $\sigma_{M}$, is obtained by the mean of the local stresses $\sigma(i)$ across the $n$ cross-sections that contain the molecule \cite{Parkinson1997}:

\begin{equation}
	\sigma_{M} = \dfrac{1}{n}\sum_{i=1}^{n} \sigma(i).
	\label{eq3}
\end{equation}

A molecule’s fracture resistance is related to the number of intermolecular bonds it forms within the backbone \cite{Vater1979}. In an attempt to quantify this resistance for a given molecule, we calculate the number of first neighbors for each segment at every cross-section containing the molecule. These values are then summed across all relevant layers, yielding the total number of contacts the molecule makes with its neighbors $K$. The total resistance is then $K$ times $\sigma_c$, where $\sigma_c$ is the critical stress for bond rupture, which is taken as unity, assuming the fibril is composed of identical molecules.

Finally, we define the probability to remove a single molecule as:

\begin{equation}
	P_{R} = \left( \dfrac{\sigma_{M}}{K\sigma_c} \right)^{m},
	\label{eq4}
\end{equation}

where $m$ is a constant that is related to the Weibull modulus \cite{Parkinson1997,Jones2012}. Here, we adopt $m = 2$, characterizing a moderate level of stochasticity in fracture resistance \cite{Parkinson1997}. Fig.~\ref{fig_5} shows how the probability to remove a single molecule is calculated. The analyzed molecule is shown in blue, and the calculation is performed considering the applied force $F$. Namely, we calculate the stress $\sigma(i)$ on each segment that is part of the molecule (blue pieces). The stress on a segment $\sigma_{i}$ is a fraction of the total stress supported by each molecules belonging to this cross-section.  The total stress on a molecule $\sigma_{M}$, is the average value of the stresses over all segments comprising the molecule. Finally, we can use the probability equation above to decide whether the molecule will be removed.

\begin{figure}[ht]
	\centering
	\includegraphics[width=0.9\textwidth]{Figures/figure_5.pdf}
	\caption{The left panel shows a segment of the backbone in which each rectangular bar denotes a single molecule. For simplicity, all molecules are assumed to have only five sections for visualization purposes. The molecule of interest—highlighted in blue—has its removal probability evaluated according to Eq.~\ref{eq4}. To obtain the relevant quantities, we first calculate the tensile stress, $\sigma$, on every cross-section that cuts through that molecule. For illustrative clarity, we single out the first such section: it is depicted as a semi-transparent plane slicing through the blue block, while the red dashed contour traces the exact line where the plane intersects the fibril. The right panel demonstrates the arrangement of the selected molecule in the cut plane; under the applied force $F$, the stress is assumed to be distributed uniformly over the nine constituent segments, giving $\sigma=\frac{F}{9}$. The blue molecule forms three bonds with first-neighbor units, thereby fixing its coordination number. Repeating the same analysis for each cross-section that intersects the blue molecule yields the complete set of local values required for the determination of $\sigma_M$ and $K$.}
	\label{fig_5}
\end{figure}

The removal probability $P_{R}$ is then calculated from Eq.~\ref{eq4} for each molecule and compared with a random number between $0$ and $1$ associated with each molecule; if this random number is less than $P_{R}$, the molecule is removed. This procedure is then repeated for all molecules in the aggregate. After the first sweep, with the force $F$ held constant, the removal probability $P_{R}$ is recalculated for the remaining molecules, new random numbers are generated, and the system is examined for additional breakages. This process is repeated iteratively until no more breakages occur. The force is subsequently increased by half a unit, and the procedure is repeated. Fibril rupture is defined as the occurrence of a completely void cross-section, signifying the failure of the load-bearing skeleton. These steps in the removal process are schematized in Fig.~\ref{fig_6}, where in $(A)$ the fibril is under stress, in $(B)$ some molecules (blocks) have already been severed, and in $(C)$ the rupture limit of the fibril was reached in the damaged region highlighted by the vertical solid lines. Throughout the process, the molecules removed at each applied force level were recorded.

\begin{figure}[ht]
	\centering
	\includegraphics[width=0.9\textwidth]{Figures/figure_6.pdf}
	\caption{Schematic view of the fibril backbone and the stages of the rupture process. Panel (A) presents a $2D$ view of the backbone, with one of its ends subjected to an external force $F$ applied along the $y$-axis. Each molecule (represented by gray rectangles) has a removal probability $P_{R}$, described by Eq.~\ref{eq4}. The rupture process is evaluated statistically, molecule by molecule: when a randomly chosen value is less than the probability $P_{R}$ of a molecule, it is removed from the aggregate, as shown in (B). As the applied force increases, the backbone undergoes progressive damage, as seen in (C). When a cross-section of the backbone becomes completely empty, marked by the solid lines, the aggregate breaks completely, reaching its maximum load-bearing limit.}
	\label{fig_6}
\end{figure}

Since our rupture model is probabilistic, for each value of $T_s$ we selected $10$ distinct fibrils and ran $1,000$ independent rupture simulations on each. During these simulations, increasing force levels were applied, and at each level, we recorded the number of molecules removed from the fibril skeleton as well as those remaining attached.

To quantify the mechanical resistance of the simulated fibrils, we implemented a progressive failure model under an incrementally applied axial force \cite{Parkinson1997}. The evolution of the average fraction of removed molecules, denoted by $\varphi$, as a function of the applied force $F$, was analyzed for different values of the parameter $T_s$, as shown in Fig.~\ref{fig_7}(A). We observe that $\varphi$ increases as a function of the applied force $F$. As force is applied, intermolecular bonds between molecules are progressively broken, leading to the gradual removal of molecules and, consequently, the loss of structural integrity.

As we can see from Fig.~\ref{fig_7}(A), the force threshold for rupture increases as a function of the parameter $T_s$. However, all curves exhibit a common functional form, which can be described by the following expression:


\begin{equation}
	f(x) = (1\times10^{-3})[\exp(\beta x) -1 + x^{\alpha}],
\end{equation}

where the coefficients $\alpha$ and $\beta$  are determined from the fit to the data. First, at low force levels, the fibril aggregate shows minimal response, with few or no molecular removals. For low forces, as the applied force increases, the fraction of removed
molecules, $\varphi$, is dominated by the power-law term. This leads to accelerated and cumulative damage to the backbone, creating weaknesses that facilitate the detachment of significant molecular clusters as the force continues to rise. At a critical stage, damage propagation intensifies sharply, with the exponential term gaining prominence and dominating the growth of $\varphi$. As a result, the fibril rapidly approaches its mechanical limit and ultimately ruptures \cite{Buehler2009}. In Fig.~\ref{fig_7}(B) and Fig.~\ref{fig_7}(C), we show the behavior of the coefficients $\beta$ and $\alpha$, respectively, as a function of the logarithm of $T_s$. Both coefficients exhibit a decreasing trend, stabilizing for $T_s \geq 512$. These coefficients characterize the fibril's resistance to gradual and explosive damage. The coefficient $\alpha$ determines the rate of the power-law term, which is associated with gradual damage; lower values of $\alpha$ reduce the rate of this damage progression \cite{Veres2013}. The coefficient $\beta$, in turn, modulates the exponential term, influencing explosive damage, with smaller values of $\beta$ delaying its contribution to the rupture process \cite{Zapperi1997a,Zapperi1999}.

\begin{figure}[ht]
	\centering
	\includegraphics[width=\textwidth]{Figures/figure_7.pdf}
	\caption{(A) The average fraction of removed molecules $\varphi$ as a function of the applied force $F$ until the fibril reaches the rupture limit for $T_s = 8, 32, 128$ and  $8192$. Despite varying the force thresholds for rupture, all curves are well described by the same functional form: $f(x) = (1\times10^{-3})[\exp(\beta x) -1 + x^{\alpha}]$. The solid black line represents the best fit of $f(x)$ to the data. Panels (B) and (C) show the coefficients $\beta$ and $\alpha$  as a function of the logarithm of $T_s$. We observe a consistent behavior of the coefficients, with a decay as $T_s$ increases, stabilizing for $T_s \geq 512$.}
	\label{fig_7}
\end{figure}

During the rupture process, the removal of a molecule creates a zone of weakness within the fibril, increasing the likelihood of subsequent molecular detachments and triggering a cascade of breakages. We observe that molecules detach either individually or in clusters. Here, we define a cluster as any group of two or more adjacent molecules that are removed following the same force increment. The detachment of these clusters constitutes an avalanche-like behavior, a phenomenon commonly observed in various natural systems such as sandpile dynamics \cite{Jaeger1992}, earthquakes \cite{Godano1993}, neural networks \cite{Beggs2003}, and lung inflation \cite{Baraba1996, Alencar2001, Suki1994}. To evaluate the significance of collective, cluster-based fractures in the overall damage, we introduce the parameter $\Psi$, defined as the fraction of molecules that detach within clusters, relative to the total number of breakages observed at a given applied force. As shown in Figure.~\ref{fig_8}, which presents the average results over 10 distinct fibrils for each $T_s$ value, the evolution of $\Psi$ reveals a strong dependence on fibril structure. For fibrils formed with low $T_s$, the relevance of cluster-based fractures is high even at low applied forces, indicating an inherent structural fragility. In contrast, reinforced fibrils (high $T_s$) are more resilient; they initially resist collective failure, with the contribution from cluster detachments only becoming significant near the ultimate rupture force, where the fibril's failure is triggered by a final cascade of breaking clusters.


\begin{figure}[ht]
	\centering
	\includegraphics[width=0.7\textwidth]{Figures/figure_8.pdf}
	\caption{The average fraction of molecules removed in a cluster, $\Psi$, as a function of the normalized force $F_n$ for $T_s = 8, 32, 128$ and $8192$. For low $T_s$, cluster breakages contribute significantly even at low forces, indicating early onset of collective failure. In contrast, for high $T_s$, the fibril initially resists through isolated bond breaking, with cluster breakages (and catastrophic failure) occurring only near the ultimate force.}
	\label{fig_8}
\end{figure}

To quantitatively characterize this behavior, we define the avalanche size, $s$, as the cluster size, i.e., the number of molecules that detach under a fixed force. Using this definition, we calculated $P(s)$, the probability distribution of avalanche sizes, for each value of $T_s$. Our breakage model relies on statistically estimating the failure probability across various molecular conformations; for each tested $T_s$, we selected 10 distinct fibrils and ran $1,000$ independent rupture simulations on each. The resulting distributions are shown in the inset of Fig.~\ref{fig_9}(A) on a log-log plot for $T_s=2$ and $T_s=8192$. To characterize these distributions, we applied a scaling law \cite{Zapperi1997b}:

\begin{equation}
	P(s) \sim s^{-\gamma},
	\label{eq5}
\end{equation}
where $\gamma$ is the characteristic exponent of the power law.

The main panel in Figure.~\ref{fig_9}(A) shows the scaling region of the distributions where this power-law behavior provides a good fit. Our analysis shows that for $T_s = 2$, the exponent is $\gamma=2.31$, while for $T_s=8192$, it increases to $\gamma=2.80$. Between these two limits, the power-law exponent $\gamma$ varies with $T_s$, as depicted in the linear-log plot in Figure.~\ref{fig_9}(B). Two distinct regimes are evident: for $T_s < 512$, the exponent grows linearly, whereas for $T_s \geq 512$, it enters a plateau. This trend indicates that the avalanche size distribution becomes narrower (i.e., a higher exponent $\gamma$) as $T_s$ increases and reaches a regime of saturation, reflecting a shift in the system's mechanical response. The transition between these slopes may correspond to a change in the dominant failure mechanism due to the increased compactness of the fibril at higher values of $T_s$ \cite{Zhao2025}. A power-law distribution indicates the lack of a characteristic scale; indeed, distributions describing phenomena with a characteristic size are known to decrease exponentially. Thus, our finding of a power-law distribution for the avalanche sizes implies that the fibril rupture process is not dominated by any characteristic event size \cite{Bak1987}.

\begin{figure}[ht]
	\centering
	\includegraphics[width=\textwidth]{Figures/figure_9.pdf}
	\caption{(A) Log-log plot of the rupture avalanche distribution, $P(s)$, as a function of the avalanche size $s$ for $T_s = 2, 32, 128$ and $8192$. A power-law behavior is observed, with well-defined slopes $\gamma$ ranging from $2.31$ to $2.80$. The solid lines represent the linear regression for each data set. The inset shows the full distribution, including its tail, for $T_s = 2$ and $T_s = 8192$. Data sets are vertically shifted for clarity. (B) The exponent $\gamma$ as a function of the parameter $T_s$ on a linear-log scale. The exponent exhibits two distinct regimes: linear growth for $T_s < 512$ (solid black line) and a plateau for $T_s \geq 512$.}
	\label{fig_9}
\end{figure}

A direct consequence of avalanche-like ruptures in the fracture process should manifest as sudden stress drops in the stress-strain curve, since avalanche propagation occurs faster than the rate of strain increase \cite{Noguchi2024}. Although we cannot directly visualize this phenomenon in our model due to intrinsic limitations, experimental evidence strongly supports this prediction. Gutsmann et al. \cite{Gutsmann2004} used force spectroscopy on single microfibrils and directly observed force drops caused by intermolecular cross-link ruptures. Svensson et al. \cite{Svensson2013} reported clear stress drops in tensile tests of individual collagen fibrils, while Deng et al. \cite{Deng2024} showed similar drops in stress-strain curves. These experimental observations support the avalanche-like failure mechanism predicted by our model.

Our model is built upon significant simplifications regarding both the formation process and the subsequent mechanical behavior. In the fibrillogenesis model, collagen molecules are idealized as rigid rectangular bars, an approximation that restricts aggregation sites to the block's faces. Additionally, the molecular dynamics are simplified by confining translational diffusion to a discrete cubic lattice, thereby neglecting rotational degrees of freedom. From a physicochemical standpoint, the model also abstracts the complexity of intermolecular interactions into a single phenomenological parameter, $T_s$, which, while efficient, precludes a direct analysis of how fibril morphology would respond to variations in biochemical parameters such as pH or ionic strength \cite{Jiang2004}. A fundamental limitation of the mechanical model lies in the probabilistic nature of fracture. This methodological choice bypasses the complexity of force transmission in a disordered structure \cite{RuizFranco2022,Provenzano2006}, but it also prevents the elucidation of specific nanomechanical mechanisms like the breakage of internal bonds. Furthermore, representing the molecules as non-deformable units is another significant simplification; by not incorporating deformation, our model cannot yield stress-strain curves, and consequently, the ultimate stress and Young's modulus cannot be calculated \cite{Innocenti2022-lp}, limiting the analysis entirely to the statistics of fracture events. Despite all these simplifications, the main contribution of this work is the discovery that the rupture process is governed by a cascade of breakages exhibiting scale-free, avalanche-like behavior, and that more robust structures exhibit a higher cross-sectional fractal dimension. This work establishes a quantitative framework where the fractal dimension emerges as a key metric for mechanical resilience. It thus offers a more fundamental way to assess self-assembled biomaterials whose failure is ultimately governed by critical avalanche-like processes.


\section*{Conclusion}

In this work, we demonstrate that the surface diffusion parameter $T_s$ governs both the structural compactness and mechanical resilience of collagen fibrils through a diffusion-limited aggregation model coupled with probabilistic fracture mechanics. Our key finding is that fibril rupture proceeds via an avalanche-like failure process that follows a scale-free behavior, with the fractal dimension emerging as a quantitative predictor of mechanical strength. These results establish a framework for understanding how assembly conditions influence the failure mechanics of hierarchical biological materials.

\section*{Author Contributions}
All authors contributed equally to this work.

\section*{Acknowledgments}
We thank the Brazilian agencies CNPq, CAPES, and FUNCAP for financial support.


% Uncomment if using bibtex (default)
\bibliography{references}

% Uncomment if using biblatex
% \printbibliography

%\section*{Supplementary Material}



\end{document}
