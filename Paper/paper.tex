% \documentclass[lineno,twocolumn,endfloat,biblatex]{biophys-new}

\documentclass{biophys-new}
\usepackage[utf8]{inputenc}
\usepackage{graphicx}
\usepackage[colorlinks,allcolors=cyan!70!black]{hyperref}

% Uncomment if using biblatex
% \addbibresource{sample.bib}

\usepackage{lipsum}

\title{Scaling behaviors in simulated collagen fibrils}
\runningtitle{Biophysical Journal Template} %% For page header

\author[1]{Robert Bertoldo Tavares}
\author[2]{Michael Ferreira de Souza}
\author[3]{B\'ela Suki}
\author[1,2,*]{Ascânio Dias Araújo}
\runningauthor{Tavares et al.} %% For page header

\affil[1]{UFC, Departamento de Física, Universidade Federal do Ceará, Fortaleza, Ceará, Brasil}
\affil[2]{UFC, Departamento de Estatística e Matemática Aplicada, Universidade Federal do Ceará, Fortaleza, Ceará, Brasil}
\affil[3]{BU, Department of Biomedical Engineering, Boston University, Boston, MA, USA}
\corrauthor[*]{ascanio@fisica.ufc.br}

% \papertype{Letters}
\papertype{Article}
% \papertype{Computational Tools}


\begin{document}


\begin{frontmatter}

	\begin{abstract}
		Collagen fibrils are hierarchical protein assemblies that provide tensile strength and structural support to mammalian tissues. Understanding the relationship between assembly kinetics, structural organization, and mechanical properties is fundamental to soft matter physics and biomaterials science. Here, we investigate collagen fibril formation and mechanical failure through complementary computational models: a diffusion-limited aggregation (DLA) algorithm with lateral surface diffusion that simulates the entropy-driven self-assembly process, and a probabilistic fracture mechanics model. Fibril assembly is governed by a surface diffusion parameter $T_s$ that represents the molecular mobility and rearrangement capability of collagen molecules after initial attachment. Increasing $T_s$ drives a morphological transition from open, fractal structures (fractal dimension $D_f \approx 1.71$, characteristic of DLA) to dense, compact fibrils ($D_f \approx 1.96$, approaching Euclidean space-filling). Critically, we find that $D_f$ is a robust predictor of both tensile strength and failure mode: higher $D_f$ corresponds to enhanced load-bearing capacity through increased connectivity of the active skeleton. Mechanical failure proceeds via avalanche-like rupture events whose size distributions follow power-law scaling, with exponents that depend systematically on $T_s$. Notably, these scaling exponents exhibit two distinct linear regimes when plotted against $\log(T_s)$, indicating a transition in the dominant failure mechanism between diffusion-limited and compact packing regimes. These findings establish fractal dimension as a quantitative link between assembly conditions and mechanical resilience, providing a framework for understanding how molecular-scale processes during self-assembly determine the emergent failure mechanics of hierarchical biomaterials.
	\end{abstract}

	\begin{sigstatement}
		Understanding collagen fibril formation and damage is crucial, because collagen fibrils are fundamental to the integrity of virtually all tissues. This study reveals that the fracture process in simulated collagen fibrils is governed by avalanches and power laws with well-defined exponents. More importantly, the results demonstrate that the fractal dimension of fibril cross-section is a robust indicator of both the failure mechanics and the failure mode of collagen fibrils.
	\end{sigstatement}
\end{frontmatter}

\section*{Introduction}

Type I collagen is the principal structural protein of the mammalian extracellular matrix (ECM), providing essential mechanical integrity to tissues such as skin, tendon, bone, cartilage, and the cornea \cite{Silver2018,DavisonKotler2019,Tresoldi2013,Fedarko2014, RicoLlanos2021,Chen2015,Suki2021}. These hierarchical assemblies are critical for elasticity, tensile strength, and resilience against physical stress \cite{Silver2018,Amirrah2022}. Individual collagen molecules, characterized by a high aspect ratio ($\approx 300~\text{nm} \times 1.5~\text{nm}$) \cite{Gelse2003, Silver2018}, self-assemble into fibrils that can reach lengths of $500~\mu\text{m}$ and diameters of $500~\text{nm}$ \cite{Charvolin2019,Kadler1996, Parry1984}. These fibrils further aggregate into fibers ($1$--$20~\mu\text{m}$ diameter), forming a multiscale network capable of storing and transmitting mechanical energy \cite{Silver2008,Suki2021}.

Characterizing the mechanical behavior of collagen fibrils, particularly at the nanoscale, presents significant challenges \cite{Nalbach2022}. Experimental techniques such as atomic force microscopy (AFM) and micro-mechanical actuation have successfully probed the tensile properties of isolated fibrils. Van der Rijt et al. used AFM to measure the mechanical properties of tendon fibrils, evaluating their tensile behavior until failure \cite{vanderRijt2006}, while Svensson et al.~\cite{Svensson2018} developed a piezoelectric actuator-based technique to measure stress and strain during uniaxial stretching. However, resolving the collective molecular dynamics that govern failure remains difficult due to experimental resolution limits, particularly when characterizing small-scale forces and intermolecular interactions \cite{Andriotis2023}. Consequently, computational modeling has become indispensable for exploring the relationship between intrafibrillar organization and mechanical response under a wider range of conditions.

Several theoretical frameworks have been developed to address fibril mechanics. Buehler et al. introduced coarse-grained molecular dynamics to model large-scale deformation, while Depalle et al. investigated the role of mineralization in stabilizing fibril elasticity \cite{Buehler2006a,Buehler2006b,Buehler2008a,Buehler2008b,Depalle2016}. Other stochastic approaches have examined how enzymatic degradation alters tissue mechanics \cite{Arau2011} and how morphological features like waviness influence fiber modulus \cite{Deng2024}. Despite these advances, the influence of fractal organization on mechanical resilience and the statistical dynamics of rupture—specifically the role of failure avalanches—remains largely unexplored.

In this work, we investigate the interplay between fractal structure and failure mechanics in collagen fibrils. We employ a modified Diffusion-Limited Aggregation (DLA) model with surface diffusion to generate fibrils with tunable morphologies \cite{Parkinson1995}. By coupling this formation model with a probabilistic fracture simulation, we show that the surface diffusion parameter $T_s$ dictates both the fractal dimension and the mechanical strength of the fibril. Furthermore, we reveal that rupture defines a critical process characterized by avalanche-like failure events, which follow power-law scaling distributions reminiscent of self-organized critical systems.

\subsection*{Fibril Formation Model}

We modeled collagen fibril formation using a three-dimensional diffusion-limited aggregation (DLA) algorithm on a cubic lattice. Collagen molecules were represented as rigid rods (rectangular bars) with dimensions $L_x = 1$, $L_y = 18$, and $L_z = 1$ in lattice units (l.u.), with $L_y$ defining the characteristic length scale of the model.

The aggregation begins by placing the first molecule at the lattice center. Subsequent molecules are released one at a time from random positions on a sphere of radius $R$ centered at the origin. The radius $R$ is updated to track the maximum extent of the growing aggregate along the $y$-axis. Each molecule then performs a random walk on the lattice until one of the following conditions is met:

\begin{enumerate}
	\item The molecule attaches to the existing aggregate;
	\item The molecule is discarded if its distance from the origin exceeds $2R$.
\end{enumerate}

After each attachment, $R$ is recalculated. For simplicity, molecular rotation during diffusion is not allowed. This cycle of release, diffusion, and attachment continues until the aggregate reaches the desired size, limited here to 30,000 molecules.

Diffusion is implemented as a three-dimensional random walk on the cubic lattice. In the $x$-$z$ plane, molecules can move to first-neighbor sites ($1~\text{l.u.}$) or second-neighbor sites ($\sqrt{2}~\text{l.u.}$, i.e., diagonal moves). Along the $y$-axis, movement is restricted to first-neighbor sites ($1~\text{l.u.}$). In real collagen fibrillogenesis, lateral aggregation occurs with a characteristic axial spacing of $67~\text{nm}$ \cite{Kadler1996}, which constrains the positions where new molecules can be incorporated. In our model, molecular attachment is restricted to multiples of $4~\text{l.u.}$ along the $y$-axis \cite{Parkinson1995}.

The formation of real fibrils is driven by electrostatic forces \cite{Parkinson1995} that favor lateral association and minimize exposed surface area \cite{Kadler1987,Parkinson1995}. To mimic this relaxation, we implemented a lateral surface-diffusion step after attachment: a newly incorporated molecule performs a random walk restricted to the $x$-$z$ plane (no movement along the $y$-axis), using the same step rules described above. The molecule is allowed up to $T_s$ diffusion attempts to search the local surface and is placed at the position that minimizes its exposed area; if multiple positions are equivalent, the first one encountered is retained \cite{Garci1991}. Using this algorithm, we generated ensembles of fibrils (each containing at least 30,000 rods) for different values of $T_s$ to examine how post-attachment mobility controls fibril compactness, structure, and subsequent mechanical response.

\begin{figure}[!htb]
	\centering
	\includegraphics[width=0.7\textwidth]{Figures/figure_1.pdf}
	\caption{Representation of portions of fibrils generated with the DLA algorithm containing 30,000 molecules. Panels $(A)$ and $(B)$ show typical fibrils obtained with $T_s = 2$ (short-term diffusion) and $T_s = 8192$ (long-term diffusion), respectively. Colors indicate the relative time of attachment, from early (blue) to late (red).}
	\label{fig_1}
\end{figure}

The generated fibrils exhibit complex, heterogeneous packing with internal voids whose prevalence depends strongly on $T_s$ (Fig.~\ref{fig_1}). For $T_s = 2$, the fibril displays an open architecture with visible gaps between molecular clusters. In contrast, for $T_s = 8192$, the structure becomes more compact and homogeneous, with substantially reduced void space.

\subsection*{Radial Expansion of the Fibril Core}

Fibril growth occurs more strongly in the $y$-direction. We aimed to study the density of the cross-section in the $x$-$z$ plane. Using the original simulation for this purpose would be computationally prohibitive. Therefore, alternatively, we isolated a trunk in the $y$-direction from $-100$ to $100$ and used a cylindrical surface instead of a sphere for particle launching. We repeated the deposition until the average radius of the cross-sections ($y=-100, -99, \dots, 100$) reached $100~\text{l.u.}$. Molecules diffused inward until binding to the aggregate, following the same diffusion, attachment, and surface-diffusion rules described above. Fig.~\ref{fig_2} shows the resulting projected central segments in the $x$-$z$ plane for different values of $T_s$.

\begin{figure}[!htb]
	\centering
	\includegraphics[width=0.7\textwidth]{Figures/figure_2.pdf}
	\caption{Representative examples of projected central segments ($x$-$z$ plane) for different values of $T_s$: (a) $T_s = 2$, (b) $T_s = 64$, (c) $T_s = 512$, and (d) $T_s = 8192$. The progression illustrates a clear transition from sparse, irregular morphology at low $T_s$ to dense, radially symmetric packing at high $T_s$. The color gradient (blue to red) represents the temporal sequence of molecular incorporation, from early-attached to recently added molecules.}
	\label{fig_2}
\end{figure}

Fig.~\ref{fig_2} illustrates how $T_s$ controls radial organization. At low $T_s$ (Fig.~\ref{fig_2}a), the projected cross-section is open and branched. As $T_s$ increases, growth becomes progressively more compact and regular, reaching the most uniform configuration at $T_s = 8192$ (Fig.~\ref{fig_2}d). Because the number of intermolecular contacts increases with local packing density, this transition suggests enhanced connectivity and mechanical stability at larger $T_s$ \cite{Mohammadkhah2023}.

To quantify how $T_s$ modulates fibril structure, we estimated the fractal dimension of the central segments. For each $T_s$, we generated 50 fibrils using the first model (spherical launch), isolated the central region of each fibril, and performed radial growth using the second model (cylindrical launch). From each grown region, we extracted 11 cross-sections from $y=-90$ to $y=90$ with a spacing of $18~\text{l.u.}$, yielding a total of 550 cross-sections per $T_s$ value. For each cross-section, we computed the center of mass and the individual maximum radius enclosing all particles; $R_{\text{max}}$ was then defined as the largest of these 550 individual radii. The mass $m(R)$ was measured as the number of particles within a circular region of radius $R$ centered at the cross-section's center of mass. For each value of $R$ spanning from $R_{\text{min}} = 5~\text{l.u.}$ to $R_{\text{max}}$, we computed the mean mass $\langle m(R) \rangle$ by averaging over all 550 cross-sections \cite{Vicsek1991}. The fractal dimension $D_f$ was then extracted from the mass--radius scaling relation \cite{Giordano2012}:

\begin{equation}
	\langle m(R) \rangle \sim R^{D_{f}},
	\label{eq1}
\end{equation}

by performing a linear fit on a log-log plot of $\langle m(R) \rangle$ versus $R$.

As shown in Fig.~\ref{fig_3}, the fractal dimension $D_f$ depends strongly on $T_s$, providing a quantitative measure of the morphological transition observed in Fig.~\ref{fig_2}. For $T_s = 2$, we find $D_f = 1.708 \pm 0.005$, close to the characteristic value of two-dimensional diffusion-limited aggregation ($D_f \approx 1.71$) \cite{Witten1983}. As $T_s$ increases, $D_f$ rises and saturates at $D_f = 1.963 \pm 0.001$ for $T_s = 8192$, approaching the Euclidean limit $D_f = 2.0$ expected for a compact two-dimensional disk. Thus, increasing post-attachment mobility drives a crossover from diffusion-limited, fractal packing to nearly space-filling cross-sections.

\begin{figure}[ht]
	\centering
	\includegraphics[width=0.7\textwidth]{Figures/figure_3.pdf}
	\caption{Average fractal dimension $D_f$ as a function of the diffusion parameter $T_s$ on a linear--log scale. Error bars are shown in red. The fractal dimension increases from $D_f = 1.708 \pm 0.005$ at $T_s = 2$ to $D_f = 1.963 \pm 0.001$ for $T_s = 8192$.}
	\label{fig_3}
\end{figure}


\subsection*{Mechanical Model for Fibril Rupture}

To connect the $T_s$-dependent fibril architecture to mechanical failure, we modeled rupture using a progressive, probabilistic fracture framework commonly employed for disordered networks \cite{z2019,Gilabert1987,Noguchi2024}. Because collagen molecules are represented as rigid rods that do not undergo continuous deformation prior to failure, we do not attempt to resolve deterministic crack propagation. Instead, intermolecular bonds fail stochastically under an applied axial force, and whole molecules detach from the load-bearing network with a stress-dependent probability. The procedure consists of:
$(i)$ isolating a trunk (force-carrying core) with dimensions $S_x = 17$, $S_y = 201$, and $S_z = 17$ l.u. \cite{Parkinson1997}. Truncation inevitably creates disconnected dangling ends that do not contribute to force transmission \cite{Herrmann1984}; we therefore
$(ii)$ identify the load-bearing backbone (active skeleton) within the trunk prior to applying tensile loading \cite{Moreira2012}.

Backbone extraction proceeds in two passes. In the first pass, molecules at $y = 1$ are marked as active, and we scan upward to $y = S_y$, progressively marking molecules in contact with active ones; unmarked molecules are then removed. In the second pass, molecules at $y = S_y$ are marked as active, and we scan downward to $y = 1$ using the same connectivity rule; only molecules marked active after this pass are retained, defining the final backbone \cite{Parkinson1997} (Fig.~\ref{fig_4}).

\begin{figure}[ht]
	\centering
	\includegraphics[width=\textwidth]{Figures/figure_4.pdf}
	\caption{(A) Two-dimensional visualization of an isolated fibril trunk used to extract the load-bearing backbone. The circle highlights the region where the identification procedure starts ($y = 1$). (B) Example of the backbone identification procedure from left to right: molecules marked as active are highlighted in black as connectivity is propagated along the trunk, yielding the final load-bearing backbone.}
	\label{fig_4}
\end{figure}

With the backbone identified, we simulate axial tension by incrementally applying an external force $F$ along the main axis ($y$). At each cross-section $i$, we define the effective load-bearing area as the number of backbone segments intersecting that plane, $N(i)$. Each discrete segment is assigned unit area on the lattice \cite{Parkinson1995}; assuming uniform load sharing, the local tensile stress at cross-section $i$ is \cite{Parkinson1997}:

\begin{equation}
	\sigma(i) = \dfrac{F}{N(i)}.
	\label{eq2}
\end{equation}

We assume that failure occurs through rupture of intermolecular bonds, such that molecules detach intact rather than breaking internally \cite{Parkinson1997}. For a molecule spanning $n$ cross-sections, the mean stress is computed as the average of the local cross-sectional stresses \cite{Parkinson1997}:

\begin{equation}
	\sigma_{M} = \dfrac{1}{n}\sum_{i=1}^{n} \sigma(i).
	\label{eq3}
\end{equation}

A molecule's fracture resistance is taken to scale with the number of intermolecular contacts it forms within the backbone \cite{Vater1979}. For each molecule, we count first-neighbor contacts for every segment in each intersected cross-section and sum these contributions to obtain the total coordination number $K$. The corresponding failure threshold is $K\sigma_c$, where $\sigma_c$ is the critical stress for bond rupture; we set $\sigma_c = 1$, assuming identical molecules.

The probability of removing a molecule is then:

\begin{equation}
	P_{R} = \left( \dfrac{\sigma_{M}}{K\sigma_c} \right)^{m},
	\label{eq4}
\end{equation}

where $m$ controls the breadth of the strength distribution and is related to the Weibull modulus \cite{Parkinson1997,Jones2012}. Unless otherwise stated, we use $m = 2$ \cite{Parkinson1997}. Figure~\ref{fig_5} schematically illustrates the calculation of $\sigma(i)$, $\sigma_{M}$, and $K$ for a representative molecule.

\begin{figure}[ht]
	\centering
	\includegraphics[width=0.9\textwidth]{Figures/figure_5.pdf}
	\caption{Schematic illustration of the computation of the removal probability $P_{R}$ for a representative molecule (blue). Left: a portion of the load-bearing backbone where each rectangle represents a molecule (drawn with five segments for clarity). The blue molecule intersects multiple cross-sections; for each intersected cross-section $i$, the local stress is $\sigma(i)=F/N(i)$ (right; example with $N(i)=9$, giving $\sigma(i)=F/9$). Averaging over intersected cross-sections yields $\sigma_{M}$ (Eq.~\ref{eq3}). The coordination number $K$ is obtained from first-neighbor contacts within each cross-section and summed over the molecule. These quantities determine $P_{R}$ via Eq.~\ref{eq4}.}
	\label{fig_5}
\end{figure}

For a given force level $F$, we evaluate $P_{R}$ for every molecule in the backbone and compare it with an independent random number $u\in(0,1)$. If $u < P_{R}$, the molecule is removed. After each sweep, stresses and probabilities are recalculated for the remaining backbone, and the sweep is repeated until no additional removals occur. The applied force is then increased by $\Delta F = 0.5$, and the procedure is repeated. Rupture is defined as the appearance of a completely void cross-section, signaling failure of the load-bearing backbone. The steps of this removal process are schematized in Fig.~\ref{fig_6}. Throughout the simulation, we record the molecules removed at each force level.

\begin{figure}[ht]
	\centering
	\includegraphics[width=0.9\textwidth]{Figures/figure_6.pdf}
	\caption{Schematic view of the load-bearing backbone and the stages of the rupture process. (A) Two-dimensional view of the backbone subjected to an axial force $F$ along the $y$-axis. Each molecule (gray) is assigned a removal probability $P_{R}$ (Eq.~\ref{eq4}); a molecule is removed when a random number $u\in(0,1)$ satisfies $u<P_{R}$. (B) Progressive damage after some molecules have been removed. (C) Rupture occurs when a cross-section becomes completely empty (solid lines), eliminating the continuous load path.}
	\label{fig_6}
\end{figure}

Since our rupture model is probabilistic, for each value of $T_s$ we analyzed an ensemble of $10$ distinct fibrils and performed $1,000$ independent rupture simulations per fibril. During each simulation, the force was increased incrementally, and at each level we recorded the fraction of molecules removed from the backbone.

To quantify the mechanical resistance of the simulated fibrils, we analyzed the evolution of the average fraction of removed molecules, denoted by $\varphi$, as a function of the applied force $F$ \cite{Parkinson1997}. As shown in Fig.~\ref{fig_7}(A), $\varphi$ increases monotonically with $F$, reflecting progressive intermolecular bond rupture and loss of load-bearing connectivity.

The force threshold for rupture increases with $T_s$, consistent with enhanced connectivity in more compact fibrils. Despite different rupture thresholds, all curves exhibit a common functional form:


\begin{equation}
	f(F) = 10^{-3}\left[\exp(\beta F) -1 + F^{\alpha}\right],
\end{equation}

where the coefficients $\alpha$ and $\beta$ are determined from fits to the data. At low forces, the response is dominated by the power-law term, corresponding to gradual damage accumulation. Close to rupture, the exponential term dominates, producing a rapid increase in $\varphi$ \cite{Buehler2009}. Figures~\ref{fig_7}(B,C) show that both $\alpha$ and $\beta$ decrease with $T_s$ and stabilize for $T_s \geq 512$. The parameter $\alpha$ sets the rate of gradual (power-law) damage accumulation \cite{Veres2013}, whereas $\beta$ controls the onset of rapid, avalanche-driven failure \cite{Zapperi1997a,Zapperi1999}.

\begin{figure}[ht]
	\centering
	\includegraphics[width=\textwidth]{Figures/figure_7.pdf}
	\caption{(A) The average fraction of removed molecules $\varphi$ as a function of the applied force $F$ until rupture for $T_s = 8, 32, 128$ and $8192$. Despite varying rupture thresholds, all curves are well described by the same functional form: $f(F) = 10^{-3}\left[\exp(\beta F) -1 + F^{\alpha}\right]$. The solid black line represents the best fit. Panels (B) and (C) show the fitted coefficients $\beta$ and $\alpha$ as a function of $\log(T_s)$, which decrease with $T_s$ and stabilize for $T_s \geq 512$.}
	\label{fig_7}
\end{figure}

During rupture, molecules detach either individually or in spatially connected clusters removed during the same force increment. We define a cluster as any group of two or more adjacent molecules removed at the same force step. Such cluster detachments correspond to avalanches, a hallmark of crackling dynamics in driven disordered systems \cite{Jaeger1992,Godano1993,Beggs2003,Baraba1996,Alencar2001,Suki1994}. To quantify the importance of collective events, we introduce $\Psi$, defined as the fraction of removed molecules that belong to clusters relative to the total number removed at a given force level. To compare across different rupture forces, we express the force as a normalized variable $F_n = F/F_{\mathrm{rup}}$, where $F_{\mathrm{rup}}$ is the rupture force of each realization. As shown in Fig.~\ref{fig_8}, low-$T_s$ fibrils exhibit large $\Psi$ even at small $F_n$, indicating early collective failure, whereas high-$T_s$ fibrils primarily fail through isolated removals until close to rupture, when $\Psi$ rises sharply.


\begin{figure}[ht]
	\centering
	\includegraphics[width=0.7\textwidth]{Figures/figure_8.pdf}
	\caption{The average fraction of molecules removed in a cluster, $\Psi$, as a function of normalized force $F_n = F/F_{\mathrm{rup}}$ for $T_s = 8, 32, 128$ and $8192$. For low $T_s$, cluster breakages contribute significantly even at low forces, indicating early onset of collective failure. In contrast, for high $T_s$, the fibril initially resists through isolated bond breaking, with cluster breakages (and catastrophic failure) occurring only near rupture.}
	\label{fig_8}
\end{figure}

To quantitatively characterize avalanche statistics, we define the avalanche size, $s$, as the cluster size, i.e., the number of molecules in a removed cluster at a fixed force level. We then compute $P(s)$, the probability distribution of avalanche sizes, for each value of $T_s$. The inset of Fig.~\ref{fig_9}(A) shows representative distributions for $T_s=2$ and $T_s=8192$. To characterize these distributions, we apply a scaling law \cite{Zapperi1997b}:

\begin{equation}
	P(s) \sim s^{-\gamma},
	\label{eq5}
\end{equation}
where $\gamma$ is the characteristic exponent of the power law.

The main panel in Fig.~\ref{fig_9}(A) shows the scaling regime where this power-law behavior provides a good fit. Our analysis yields $\gamma=2.31$ for $T_s = 2$ and $\gamma=2.80$ for $T_s=8192$. Between these limits, $\gamma$ varies systematically with $T_s$ (Fig.~\ref{fig_9}B): it increases approximately linearly with $\log(T_s)$ for $T_s < 512$ and reaches a plateau for $T_s \geq 512$. Increasing $\gamma$ indicates a narrowing of the avalanche-size distribution, consistent with a transition in the dominant failure mechanism as fibrils become more compact at larger $T_s$ \cite{Zhao2025}. The persistence of a power-law tail indicates scale-free fracture dynamics without a characteristic avalanche size \cite{Bak1987}.

\begin{figure}[ht]
	\centering
	\includegraphics[width=\textwidth]{Figures/figure_9.pdf}
	\caption{(A) Log--log plot of the rupture avalanche distribution, $P(s)$, as a function of avalanche size $s$ for $T_s = 2, 32, 128$ and $8192$. A power-law regime is observed, with slopes $\gamma$ ranging from $2.31$ to $2.80$ (solid lines: linear regressions). The inset shows the full distributions, including the tails, for $T_s = 2$ and $T_s = 8192$. Data sets are vertically shifted for clarity. (B) The exponent $\gamma$ as a function of $T_s$ on a linear--log scale, showing two regimes: growth for $T_s < 512$ and a plateau for $T_s \geq 512$.}
	\label{fig_9}
\end{figure}

A direct consequence of avalanche-like rupture should be sudden stress drops in stress--strain curves, since avalanches propagate faster than the imposed loading rate \cite{Noguchi2024}. Although our model does not resolve continuous strain and thus cannot generate stress--strain curves, experimental evidence strongly supports this prediction. Gutsmann et al. \cite{Gutsmann2004} used force spectroscopy on single microfibrils and directly observed force drops caused by intermolecular cross-link rupture. Svensson et al. \cite{Svensson2013} reported clear stress drops in tensile tests of individual collagen fibrils, while Deng et al. \cite{Deng2024} observed similar drops in stress--strain curves.

Our framework is intentionally coarse-grained. In the assembly model, collagen molecules are idealized as rigid lattice rods, restricting aggregation sites to the rod faces and neglecting rotational diffusion. Intermolecular interactions are condensed into a single phenomenological parameter, $T_s$, which precludes a direct mapping to biochemical conditions such as pH or ionic strength \cite{Jiang2004}. In the mechanical model, fracture is implemented probabilistically, bypassing detailed force transmission and specific nanomechanical bond-breaking mechanisms \cite{RuizFranco2022,Provenzano2006}. Moreover, because molecules are non-deformable, the model cannot produce stress--strain curves or elastic moduli \cite{Innocenti2022-lp}. Despite these limitations, the simulations reveal that rupture proceeds through scale-free, avalanche-like dynamics and that increasing structural compactness (higher cross-sectional fractal dimension) correlates with enhanced mechanical resilience. This connection suggests that the fractal dimension provides a quantitative metric to relate assembly conditions to failure statistics in self-assembled biomaterials.


\section*{Conclusion}

In this work, we demonstrate that the surface diffusion parameter $T_s$ governs both the structural compactness and mechanical resilience of collagen fibrils through a diffusion-limited aggregation model coupled with probabilistic fracture mechanics. Our key finding is that fibril rupture proceeds via an avalanche-like failure process that follows a scale-free behavior, with the fractal dimension emerging as a quantitative predictor of mechanical strength. These results establish a framework for understanding how assembly conditions influence the failure mechanics of hierarchical biological materials.

\section*{Author Contributions}
All authors contributed equally to this work.

\section*{Acknowledgments}
We thank the Brazilian agencies CNPq, CAPES, and FUNCAP for financial support.


% Uncomment if using bibtex (default)
\bibliography{references}

% Uncomment if using biblatex
% \printbibliography

%\section*{Supplementary Material}



\end{document}
